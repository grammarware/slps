\chapter{TXL}

 Source name: \textbf{txl}

\section{Source grammar}

\begin{itemize}
\item Source artifact: \href{http://github.com/grammarware/slps/blob/master/topics/fl/txl/FL.Txl}{topics/fl/txl/FL.Txl}
\item Grammar extractor: \href{http://github.com/grammarware/slps/blob/master/topics/extraction/txl/txl2bgf.xslt}{topics/extraction/txl/txl2bgf.xslt}
\end{itemize}

\footnotesize\begin{center}\begin{tabular}{|l|}\hline
\multicolumn{1}{|>{\columncolor[gray]{.9}}c|}{\footnotesize \textbf{Production rules}}
\\\hline
$\mathrm{p}(\text{`'},\mathit{program},\plus \left(\mathit{fun}\right))$	\\
$\mathrm{p}(\text{`'},\mathit{fun},\mathrm{seq}\left(\left[\mathit{id}, \plus \left(\mathit{id}\right), \text{`='}, \mathit{expression}, \mathit{newline}\right]\right))$	\\
$\mathrm{p}(\text{`'},\mathit{expression},\mathrm{choice}([\mathrm{seq}\left(\left[\mathit{expression}, \mathit{op}, \mathit{expression}\right]\right)$\\$\qquad\qquad\mathrm{seq}\left(\left[\mathit{id}, \plus \left(\mathit{expression}\right)\right]\right)$\\$\qquad\qquad\mathrm{seq}\left(\left[\text{`if'}, \mathit{expression}, \text{`then'}, \mathit{expression}, \text{`else'}, \mathit{expression}\right]\right)$\\$\qquad\qquad\mathrm{seq}\left(\left[\text{`('}, \mathit{expression}, \text{`)'}\right]\right)$\\$\qquad\qquad\mathit{id}$\\$\qquad\qquad\mathit{number}]))$	\\
$\mathrm{p}(\text{`'},\mathit{op},\mathrm{choice}([\text{`+'}$\\$\qquad\qquad\text{`-'}$\\$\qquad\qquad\text{`=='}]))$	\\
\hline\end{tabular}\end{center}



\section{Normalizations}
{\footnotesize\begin{itemize}
\item \textbf{abstractize-concretize}\\$\mathrm{p}\left(\text{`'},\mathit{fun},\mathrm{seq}\left(\left[\mathit{id}, \plus \left(\mathit{id}\right), \fbox{$\text{`='}$}, \mathit{expression}, \mathit{newline}\right]\right)\right)$
\item \textbf{abstractize-concretize}\\$\mathrm{p}\left(\text{`'},\mathit{op},\mathrm{choice}\left(\left[\fbox{$\text{`+'}$}, \fbox{$\text{`-'}$}, \fbox{$\text{`=='}$}\right]\right)\right)$
\item \textbf{abstractize-concretize}\\$\mathrm{p}\left(\text{`'},\mathit{expression},\mathrm{choice}\left(\left[\mathrm{seq}\left(\left[\mathit{expression}, \mathit{op}, \mathit{expression}\right]\right), \mathrm{seq}\left(\left[\mathit{id}, \plus \left(\mathit{expression}\right)\right]\right), \mathrm{seq}\left(\left[\fbox{$\text{`if'}$}, \mathit{expression}, \fbox{$\text{`then'}$}, \mathit{expression}, \fbox{$\text{`else'}$}, \mathit{expression}\right]\right), \mathrm{seq}\left(\left[\fbox{$\text{`('}$}, \mathit{expression}, \fbox{$\text{`)'}$}\right]\right), \mathit{id}, \mathit{number}\right]\right)\right)$
\item \textbf{vertical-horizontal}  in $\mathit{expression}$
\item \textbf{undefine-define}\\$\mathrm{p}\left(\text{`'},\mathit{op},\varepsilon\right)$
\item \textbf{abridge-detour}\\$\mathrm{p}\left(\text{`'},\mathit{expression},\mathit{expression}\right)$
\item \textbf{extract-inline}  in $\mathit{expression}$\\$\mathrm{p}\left(\text{`'},\mathit{expression_1},\mathrm{seq}\left(\left[\mathit{expression}, \mathit{op}, \mathit{expression}\right]\right)\right)$
\item \textbf{extract-inline}  in $\mathit{expression}$\\$\mathrm{p}\left(\text{`'},\mathit{expression_2},\mathrm{seq}\left(\left[\mathit{id}, \plus \left(\mathit{expression}\right)\right]\right)\right)$
\item \textbf{extract-inline}  in $\mathit{expression}$\\$\mathrm{p}\left(\text{`'},\mathit{expression_3},\mathrm{seq}\left(\left[\mathit{expression}, \mathit{expression}, \mathit{expression}\right]\right)\right)$
\end{itemize}}

\section{Grammar in ANF}

\footnotesize\begin{center}\begin{tabular}{|l|c|}\hline
\multicolumn{1}{|>{\columncolor[gray]{.9}}c|}{\footnotesize \textbf{Production rule}} &
\multicolumn{1}{>{\columncolor[gray]{.9}}c|}{\footnotesize \textbf{Production signature}}
\\\hline
$\mathrm{p}\left(\text{`'},\mathit{program},\plus \left(\mathit{fun}\right)\right)$	&	$\{ \langle \mathit{fun}, {+}\rangle\}$\\
$\mathrm{p}\left(\text{`'},\mathit{fun},\mathrm{seq}\left(\left[\mathit{id}, \plus \left(\mathit{id}\right), \mathit{expression}, \mathit{newline}\right]\right)\right)$	&	$\{ \langle \mathit{newline}, 1\rangle, \langle \mathit{id}, 1{+}\rangle, \langle \mathit{expression}, 1\rangle\}$\\
$\mathrm{p}\left(\text{`'},\mathit{expression},\mathit{expression_1}\right)$	&	$\{ \langle \mathit{expression_1}, 1\rangle\}$\\
$\mathrm{p}\left(\text{`'},\mathit{expression},\mathit{expression_2}\right)$	&	$\{ \langle \mathit{expression_2}, 1\rangle\}$\\
$\mathrm{p}\left(\text{`'},\mathit{expression},\mathit{expression_3}\right)$	&	$\{ \langle \mathit{expression_3}, 1\rangle\}$\\
$\mathrm{p}\left(\text{`'},\mathit{expression},\mathit{id}\right)$	&	$\{ \langle \mathit{id}, 1\rangle\}$\\
$\mathrm{p}\left(\text{`'},\mathit{expression},\mathit{number}\right)$	&	$\{ \langle \mathit{number}, 1\rangle\}$\\
$\mathrm{p}\left(\text{`'},\mathit{expression_1},\mathrm{seq}\left(\left[\mathit{expression}, \mathit{op}, \mathit{expression}\right]\right)\right)$	&	$\{ \langle \mathit{op}, 1\rangle, \langle \mathit{expression}, 11\rangle\}$\\
$\mathrm{p}\left(\text{`'},\mathit{expression_2},\mathrm{seq}\left(\left[\mathit{id}, \plus \left(\mathit{expression}\right)\right]\right)\right)$	&	$\{ \langle \mathit{expression}, {+}\rangle, \langle \mathit{id}, 1\rangle\}$\\
$\mathrm{p}\left(\text{`'},\mathit{expression_3},\mathrm{seq}\left(\left[\mathit{expression}, \mathit{expression}, \mathit{expression}\right]\right)\right)$	&	$\{ \langle \mathit{expression}, 111\rangle\}$\\
\hline\end{tabular}\end{center}

\section{Nominal resolution}

Production rules are matched as follows (ANF on the left, master grammar on the right):
\begin{eqnarray*}
\mathrm{p}\left(\text{`'},\mathit{program},\plus \left(\mathit{fun}\right)\right) & \bumpeq & \mathrm{p}\left(\text{`'},\mathit{program},\plus \left(\mathit{function}\right)\right) \\
\mathrm{p}\left(\text{`'},\mathit{fun},\mathrm{seq}\left(\left[\mathit{id}, \plus \left(\mathit{id}\right), \mathit{expression}, \mathit{newline}\right]\right)\right) & \Bumpeq & \mathrm{p}\left(\text{`'},\mathit{function},\mathrm{seq}\left(\left[str, \plus \left(str\right), \mathit{expression}\right]\right)\right) \\
\mathrm{p}\left(\text{`'},\mathit{expression},\mathit{expression_1}\right) & \bumpeq & \mathrm{p}\left(\text{`'},\mathit{expression},\mathit{binary}\right) \\
\mathrm{p}\left(\text{`'},\mathit{expression},\mathit{expression_2}\right) & \bumpeq & \mathrm{p}\left(\text{`'},\mathit{expression},\mathit{apply}\right) \\
\mathrm{p}\left(\text{`'},\mathit{expression},\mathit{expression_3}\right) & \bumpeq & \mathrm{p}\left(\text{`'},\mathit{expression},\mathit{conditional}\right) \\
\mathrm{p}\left(\text{`'},\mathit{expression},\mathit{id}\right) & \bumpeq & \mathrm{p}\left(\text{`'},\mathit{expression},str\right) \\
\mathrm{p}\left(\text{`'},\mathit{expression},\mathit{number}\right) & \bumpeq & \mathrm{p}\left(\text{`'},\mathit{expression},int\right) \\
\mathrm{p}\left(\text{`'},\mathit{expression_1},\mathrm{seq}\left(\left[\mathit{expression}, \mathit{op}, \mathit{expression}\right]\right)\right) & \bumpeq & \mathrm{p}\left(\text{`'},\mathit{binary},\mathrm{seq}\left(\left[\mathit{expression}, \mathit{operator}, \mathit{expression}\right]\right)\right) \\
\mathrm{p}\left(\text{`'},\mathit{expression_2},\mathrm{seq}\left(\left[\mathit{id}, \plus \left(\mathit{expression}\right)\right]\right)\right) & \bumpeq & \mathrm{p}\left(\text{`'},\mathit{apply},\mathrm{seq}\left(\left[str, \plus \left(\mathit{expression}\right)\right]\right)\right) \\
\mathrm{p}\left(\text{`'},\mathit{expression_3},\mathrm{seq}\left(\left[\mathit{expression}, \mathit{expression}, \mathit{expression}\right]\right)\right) & \bumpeq & \mathrm{p}\left(\text{`'},\mathit{conditional},\mathrm{seq}\left(\left[\mathit{expression}, \mathit{expression}, \mathit{expression}\right]\right)\right) \\
\end{eqnarray*}
This yields the following nominal mapping:
\begin{align*}\mathit{txl} \:\diamond\: \mathit{master} =& \{\langle \mathit{program},\mathit{program}\rangle,\\
 & \langle \mathit{apply},\mathit{expression_2}\rangle,\\
 & \langle \mathit{function},\mathit{fun}\rangle,\\
 & \langle \mathit{expression},\mathit{expression}\rangle,\\
 & \langle str,\mathit{id}\rangle,\\
 & \langle \mathit{binary},\mathit{expression_1}\rangle,\\
 & \langle \mathit{operator},\mathit{op}\rangle,\\
 & \langle int,\mathit{number}\rangle,\\
 & \langle \omega,\mathit{newline}\rangle,\\
 & \langle \mathit{conditional},\mathit{expression_3}\rangle\}\end{align*}
 Which is exercised with these grammar transformation steps:

{\footnotesize\begin{itemize}
\item \textbf{renameN-renameN} $\mathit{expression_2}$ to $\mathit{apply}$
\item \textbf{renameN-renameN} $\mathit{fun}$ to $\mathit{function}$
\item \textbf{renameN-renameN} $\mathit{id}$ to $str$
\item \textbf{renameN-renameN} $\mathit{expression_1}$ to $\mathit{binary}$
\item \textbf{renameN-renameN} $\mathit{op}$ to $\mathit{operator}$
\item \textbf{renameN-renameN} $\mathit{number}$ to $int$
\item \textbf{renameN-renameN} $\mathit{expression_3}$ to $\mathit{conditional}$
\end{itemize}}

\section{Structural resolution}
{\footnotesize\begin{itemize}
\item \textbf{project-inject}\\$\mathrm{p}\left(\text{`'},\mathit{function},\mathrm{seq}\left(\left[str, \plus \left(str\right), \mathit{expression}, \fbox{$\mathit{newline}$}\right]\right)\right)$
\end{itemize}}
